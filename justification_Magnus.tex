\section{Justification of premises (Magnus)}

\subsection{Lives are equally important}
The notion of human lives being of equal value, or equally important, is pervasive in modern Western culture. It occurs in the first sentence of the United States Declaration of Independence \cite{Declaration}: \textit{"We hold these truths to be self-evident, that all men are created equal [...]"}
 
The idea is also present in a document passed by by France's National Constituent Assembly in August 1789,  relating to the french revolution (\cite{France}). Both of these declarations relied on the works of Enlightenment thinkers like Montesquieu. Later in 1948, the United Nations included this notion in The Universal Declaration of Human Rights \cite{UN}: \textit{"All human beings are born free and equal in dignity and rights."}

The ethical theory of Kant agrees with this view. According to Kant, an act is morally acceptable if, and only if, the principle behind the act, by Kant called the maxim, is universalizable (\cite{Schafer}, chapter 11). By “universalizable” we mean: if everyone else acted in accordance with the principle behind the act, would we accept such a world?  

In our case, the premise “Lives are equally important” is such a maxim. It is a guiding principle of actions. Kant’s idea is often condensed into a single phrase, called Kant’s categorical imperative: “Act as if the maxims of your action were to become, through your will, a universal law of nature” Our maxim passes the universalizable test – we readily accept a world in the actions of people are on a notion of equality.

The ethical theory consequentialism is concerned with doing as much good as possible.
In utilitarianism, the most popular branch of consequentialism, an action is morally required if it produces the best overall results, given the options. But this does not necessarily imply equality. According to John Stuart Mill, one of the founders of utilitarianism, actions should create “the greatest good for the greatest number” (\cite{Schafer}, chapter 9). This “greatest good” should be measured in some quantity of welfare. Alternatively, given two actions that decrease overall welfare, choose the action with the smallest decrease.

To see that utilitarianism does not necessarily imply equality, imagine a class room with 13 students. If mocking one ginger leads to the ginger losing 100 welfare (measured in some units), but his twelve non-ginger classmates gain 120 welfare (+10 welfare for each), the overall welfare is increased, so the action is morally right according to utilitarianism. But the students are not treated equally. 

\subsection{Lose the least number of lives}
The argument “Given the unavoidable choice of killing, kill as few as possible” is troublesome to Kant’s ethical theory. His categorical imperative exists in a second, equally important, version: “always treat a human being as an end, never only as
means to an end”. But by killing 1 passenger to save ten bystanders, the passenger is treated as a means to saving 10 lives. What should the car do, then? Swerve and kill the 10 passengers? No, says the imperative, now you the treat the bystanders as a means.
Both actions – killing the passenger or killing the bystanders – is morally wrong. There is no right choice.

Utilitarianism has a different answer. If the 1 passenger and 10 bystanders are roughly comparable (same age, same potential welfare gain for the rest of their lives), the world is better off with the 10 bystanders surviving, so the car should take the action killing the passenger. In that case, the problem reduces to a trolley problem\footnote{A popular thought experiment in ethics. It roughly goes: a trolley is heading to kill X persons. Should you pull the lever to divert the runaway trolley onto the side track, killing Y persons?}.

In our problem, the welfare gain is even greater (or rather, the welfare loss smaller) if the passenger is an old, sick, and depressed. His lost welfare has a negligible impact on the overall welfare of all people. While this may seem eerie and cynical, utilitarianism \textit{does} provide an answer for what to do. 

