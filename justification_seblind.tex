\section{All lives are equally important}
\subsection{A counter argument to the counter argument}
Intuitively the premise that all lives are equal is subject to discussion --
now suppose that all lives are not equal and the car determine which lives
are of higher value than others. In the event of an unavoidable crash it
computes the outcome with highest overall value of welfare and acts accordingly.
This behavior follows utilitarianism, but requires a well-defined system of
how humanity decides to award value to lives. Which quantities/variables should be
considered in such a system. Age? criminal records?
Will the associated quality of life to a person at the age of 90 with only 3 years left to live,
have less value than that of teenager with a long life ahead? Given only the age, maybe, but if we
know that the teenager is depressed and suicidal, maybe not.

According to Kant's categorical imperative, the maxim (here the act of associating variables to welfare),
which is shown from the example above to be an incomplete picture of welfare,
would constitute a law impossible to universalize.
Actual welfare calculation is too complex since welfare in itself is a quantity
that requires all-knowing understanding, that means beyond human level.
On that account it is only right to evaluate all lives equally.
Hypothetically, if the implementation did have all-knowing understanding, then we
allow the designers to choose who lives or dies effectively promoting them to
a God-like status. At this point humanity needs to justify that judgements of such magnitude,
carried out by a small group of individuals, are deemed to be fair.
Once again, the compromise not to differentiate seems like the morally right approach.
