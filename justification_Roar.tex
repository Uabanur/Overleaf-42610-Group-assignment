\section{Justification of premises (Roar)}

\textbf{All lives are equally important.}
According to the ethical theory consequentialism, \textit{"An action is right if and only if there is no other action in your set of options that will lead to better agent-neutral consequences, all things considered."} (cite slides: Ethical Theories, MMA)

Since consequentialism acts agent-neutrally, all agents are intrinsically equal, and by extension their lives (all else being equal) are equal.

In utilitarianism, the most popular branch of consequentialism, these "consequences" are represented by some metric "welfare" (that is yet to defined). 

At first utilitarianism does not seem to help us with this premise. Since the lives are weighted by their individual contribution and impact on the total welfare, they are not necessarily assumed equal. This gives rise to a new dilemma, as to how we holistically calculate the total welfare each person is connected to. In the scenario of an unavoidable crash, an immediate decision must be taken. It seems impossible for the vehicles software to take everything, from every person, into consideration when choosing between which lives to keep and which lives to sacrifice. The welfare connecting to each view/perspective of a person is an immeasurable quantity. The outcome therefore cannot be claimed to be the absolute optimal option. Since choosing between lives is an unfeasible task with respect to the maximizing the sum of welfare, we should not evaluate lives on those terms, and lives should therefore be handled equally.

Slippery slope arguments:
Should we allow differentiating between lives?
Are we allowed to choose who dies?

Kant?

Contractarianism = \textit{"An action is right if and only if it does not violate a contract that free, equal, and rational people would agree to live by, on the condition that others obey the contract too."}

Since the mentality of contractarianism, is to evaluate your actions based on the norm of the society of free, equal, and rational people, the individuals intrinsically have an equal say. 


Based on the above analysis, it seems clear, that equality between lives and people follow what most philosophers find moral and ethical correct.



\textbf{Least amount of deaths}
To argument for this premise, we may first argument that killing innocents is wrong.

It is clear, that contractarianism and kantianism justifies this. No free, equal and rational people would agree to live by a contract legalizing the killing and innocents, and if the maxim of killing innocents were universalized (no matter the intent), society would come to an end. If killing innocents were justified in any circumstance, this would be a very slippery slope argument. 

When the choice is, that either the safety of the passenger in the car ($N_1$) or some bystanders ($N_2$), it cannot be justified that the bystanders should be sacrificed for the safety of the passenger. Assuming lives are equally important, and killing innocents is morally wrong, the least amount of fatality is optimal. If all options leads to $N_1$ amount of deaths, this is er tragic certain. The essence of premise 2, is that the remaining $N_2 - N_1$ people, should not be killed, for above stated reasons.


\textbf{PP: Passengers vs bystanders}
Given a hypothetical scenario: A car is driving towards a wall, if it goes straight the passenger has a high risk of fatality. If the car swerves, it may hit a cluster of $N_2$ people, which acts as a cushion, decelerating the car slower and therefore lowering the risk of fatal injury for the passenger.

If the car followed the ethical principle of principle PP. The car would choose to swerve with the high risk of killing the bystanders, in order to maximize the safety of the passenger.

\textbf{Cars implemented by Principle PP implications/consequences. }
If it was legalized in the European Union to sell, import and use driverless vehicles programmed to navigate in accordance with only principle PP in the event of unavoidable crash, actions such as the scenario in premise 3 would be legalized. 



\textbf{Conclusion}
From premise 1 and 2, it is shown ethically wrong to cause unnecessary deaths.

From premise 3 and 4, it is shown that vehicles programmed to navigate in accordance with only principle PP in the event of unavoidable crash would cause unnecessary deaths.

The politicians suggestion is therefore ethically wrong.
It should be \textbf{illegal} in the European Union to sell, import and use driverless vehicles programmed to navigate in accordance with only principle PP in the event of unavoidable crash.